\section{Conclusões do projeto}


\section{Trabalhos futuros}
O presente projeto pode ser encarado como uma experiência de sucesso, o objetivo
geral de se obter um modelo monofásico com baixo número de parâmetros capaz de
ser validade através do ensaio de TGA foi atingido, bem como os objetivos
específicos. Por outro lado, a baixa precisão quando comparado aos resultados
experimentais revelam a existência de novos desenvolvimentos que no contexto do
atual projeto não foram abordados. São eles:

\begin{itemize}
    \item Medidas experimentais das curva de sorção isotérmica;
    \item Propor um método de problema-inverso para obtenção da estimativa das
      curvas de sorção;
    \item Consideração de diversas camadas de materiais distintos;
    \item Implementação da formulação variacional para o problema mecânico;
    \item Realização de ensaios de TGA com amostras de diferentes tamanhos para
      considerar o efeito do volume da amostra;
    \item Reprodução dos ensaios de PTS (amostra aquecida unidimensionalmente com
      sensores de pressão, temperatura e medida da massa da amostra) e
      comparação com os dados de TGA;
    \item Estudo da presença de aditivos como fibras poliméricas e sua simulaçaõ
      em mesoescala;
    \item Acoplamento da um sistema controlador PID para a obtenção das curvas
      de sorção otimizadas para a redução da pressão gerada durante a secagem.
\end{itemize}
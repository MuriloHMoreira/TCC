\section{Conclusões do projeto}
O presente trabalho tinha como objetivo o desenvolvimento e a implementação de
um modelo númerico simples, no sentido de que capturasse os principais fenômenos
físicos que ocorrem durante a secagem de materiais monolíticos refratários, e
que fosse capaz de ser validado a partir de ensaios baratos como a TGA. Embora
o modelo utilizado seja derivado de trabalhos desenvolvidos na década de 70, é
possível observar que o primeiro objetivo foi contemplado. Porém, a
maior contribuição do projeto está na proposta de trabalho, que ao invés de
utilizar do ensaio de TGA como uma ferramenta de validação (o que fora proposto
como objetivo), passa a observar tal ensaio como uma ferramenta para a obtenção
das curvas de sorção isotérmica (seja manualmente ou através da solução do
problema inverso).

Assim, a comparação dos dados experimentais, mostrou que o modelo apresentou uma
certa coerência qualitativa com os dados experimentais, apresentando o material
aquecido na condição que gerou explosão, com
maior pressão e com níveis condizentes com uma estimativa da resistência a
tração simples do material.

Além destas contribuições, o trabalho mostrou a possibilidade de se otimizar uma
curva de secagem a partir do uso de um controlador PID. Embora houvesse um {\it
  overshoot} de 33.3\% da pressão máxima em comparação a pressão limite, tal
implementação já apresenta um forte apelo comercial e de aplicação, sendo
possível propor um {\it workflow} para o estudo de otimização de curvas de
secagem para diferentes sistemas e distintas geometrias. Tal linha de trabalho
pode ser resumida nas etapas:

\begin{itemize}
  \item Medidas das propriedades térmicas do material (condutividade térmica e
    calor específico), densidade, permeabilidade, água de desidratação e
    propriedades mecânicas do material;
  \item Calibração das curvas de sorção isotérmica a partir de um ensaio de TGA;
  \item Validação do modelo através de um ensaio de aquecimento unilateral
    (ensaio PTM);
  \item Uso do controlador PID a partir da pressão limite estimada a partir da
    resistência mecânica do material, em uma simulação da geometria final da
    peça a ser seca.
\end{itemize}

Dessa forma, seria possível obter uma solução orientada a cada contexto
permitindo o uso de curvas de secagem mais arrojadas, com menores impactos
ambientais, de menor custo associado e principalmente, com menores riscos de
explosão ou trincamento da peça.

\section{Trabalhos futuros}
O presente projeto pode ser encarado como uma experiência de sucesso, o objetivo
geral de se obter um modelo monofásico com baixo número de parâmetros validado
com um ensaio de TGA foi parcialmente atingido, bem como os objetivos
específicos. Por outro lado, a baixa precisão quando comparado aos resultados
experimentais revelam a existência de novos desenvolvimentos que no contexto do
atual projeto não foram abordados. São eles:

\begin{itemize}
    \item Medidas experimentais das curva de sorção isotérmica;
    \item Propor um método de problema-inverso para obtenção da estimativa das
      curvas de sorção;
    \item Consideração de diversas camadas de materiais distintos;
    \item Implementação da formulação variacional para o problema mecânico;
    \item Realização de ensaios de TGA com amostras de diferentes tamanhos para
      considerar o efeito do volume da amostra;
    \item Reprodução dos ensaios de PTS (amostra aquecida unidimensionalmente com
      sensores de pressão, temperatura e medida da massa da amostra) e
      comparação com os dados de TGA;
    \item Estudo da presença de aditivos como fibras poliméricas e sua simulação
      em mesoescala;
    \item Implementação do controlador PID otimizado de forma a reduzir o {\it
        overshoot} da pressão gerada durante a secagem.
\end{itemize}


%%% Local Variables:
%%% mode: latex
%%% TeX-master: "TCC-Secagem"
%%% End:

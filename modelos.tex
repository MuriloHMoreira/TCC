\section{Modelos de Secagem}\label{modelos}
Existem inúmeros esforços para a simulação do fluxo de fluídos em meios porosos.
A secgaem de concretos refratários é apenas uma das inúmeras outras aplicações
que tal simulação compreende, havendo ainda aplicações em tópicos de geofísica
(movimentação de corpos hídricos em subsolo), modelamento de poços de petróleo,
e ainda simulações do comportamento de estruturas de concreto em situações de
incêndio (essa a aplicação mais próxima na qual grande desenvolvimento tem sido
realizado, ver \ref{sec:bazant}


Os primeiros esforços para tais modelamentos foram realizados por Aleksei Vasielivich Luikov
\cite{martynenko2010} a partir de 1929. Devido ao uso ainda incipiente de
métodos numéricos, Luikov elaborou metodologias para a obtenção de soluções
analíticas para problemas de transporte de massa e energia além de trabalhos que
propunham experimentos para validação dos modelos, avançando enormenete tal área
do conhecimento, sendo reconhecido como um dos grandes pesquisadores da área.

Os modelos propostos por Ba\v{z}ant e desenvolvido pelos autores mais recentes são
derivados da formulação de Luikov, documentada em seu Livro de 1964, ``\textit{Heat and
Mass Transfer in Capillary Porous Bodies}''
\cite{luikov1964heat}.

    Conforme descrito por Luikov, o transporte de massa dentro de uma matriz
    porosa pode se dar tanto no estado gasoso quanto no estado líquido. No
    estado gasos o fluxo se dá pelo movimento do vapor e da mistura de gases
    através da diffusão a nível molecular e a nível molar através da filtração,
    isto é, o fluxo decorrente da queda de pressão da mistura vapor-gás no
    interior do material. Já o movimento do líquido pode decorrer também por
    difusão, filtração devido ao gradiente de pressão ou ainda por absorção
    capilar.

    Os modelos subsequentes simplificam os fluxos por difusão e focam na
    descrição do fluxo por filtração através da lei de Darcy. As seções
    seguintes descrevem os modelos derivados das equações de Luikov.
    
    
    
    \subsection{Modelo de Ba\v{z}ant}\label{sec:bazant}
    Zedn\v{e}k Ba\v{z}ant é Professor de Engenharia Cívil e Engenharia de
    Materiais da universidade de Northwestern reconhecido mundialmente pelos
    seus trabalhos em  mecânica dos sólidos em estruturas de concreto. No ínicio
    dos anos 70, Ba\v{z}ant estudou o efeito de altas temperaturas em estruturas
    de concreto (cimento Portland) para aplicações nucleares \cite{bundesen2004biography}.
    
    O modelo de Ba\v{z}ant se destaca por dois grandes motivos, primeiramente foi
    vanguardista ao ser o primeiro (dos trabalhos encontrados durante a revisão
    bibliográfica do presente trabalho) que utilizou de métodos numéricos para
    resolver uma formulação simplificada das equações propostas por Luikov e
    segundo, e o segundo pelas próprias simplificações utilizadas que permitiram
    uma abordagem semi-empírica direcionada com grande foco para a aplicação em
    questão. A presente seção usará uma notação próxima à utilizada pelo autor,
    porém alguns termos serão adaptados para conferir uma maior clareza.

    A formulação de Ba\v{z}ant se inícia através da definição do fluxo mássico
    de uma única fase que representa a água líquida, o vapor e o ar dentro do
    sistema, obedecendo à da Lei de Darcy e do fluxo de Soret, e do fluxo térmico através da
    lei de Fourier e do fluxo de Dofour \cite{bazant1978thermal, bazant1978, bavzant1982, bazant1979}:
    \begin{equation}
      \label{eq:Darcy}
      \vec{J}_{Darcy} = - \rho \vec{v} = - \rho \frac{\kappa }{\nu} \nabla P 
    \end{equation}

    \begin{equation}
      \label{eq:Soret}
      \vec{J}_{Soret} = - D_s \nabla T
    \end{equation}

    \begin{equation}
      \label{eq:Fourier}
     \vec{q}_{Fourier} = - \lambda \nabla T 
    \end{equation}

    \begin{equation}
      \label{eq:Dufour}
      \vec{q}_{Dufour} = - D_d \nabla P
    \end{equation}

    A simplificação de uma única fase foi disparadamente a parcela que mais se
    desenvolveu através dos trabalhos dos autores
    subsequentes \cite{Pesavento2013, Gong1995a, Gong1991, Gawin2003, Gawin2004,
      Gawin1999, Fey2016b, Davie2006a, Davie2012a, Abdel-Rahman1996}, porém,
    Ba\v{z}ant defend e que devido à morfologia dos canais pelos quais tais fluídos passam, tal simplificação é coerente dado que o caminho livre médio das moléculas de vapor (fase que seria mais móvel, em teoria) é maior que o espaçamento entre os poros, e portanto sua
    movimentação se dá linearmente, movendo consigo a camada de moléculas
    adsorvidas na superfície interna do material [REFERENCIA LIVRO].
    
    Uma vez definido tais fluxos as primeiras simplificações são feitas, os
    fluxos de Soret ($\vec{J}_{Soret}$) e de Dufour ($\vec{q}_{Dufour}$) são considerados
    desprezíveis devido resultados de estudos experimentais \cite{bazant1979}.

    Em seguida, tais fluxos são considerados em uma equação de balanço
    energético e outra de balanço mássico gerando um sistema de equações
    parciais diferenciais.

    \begin{equation}
      \label{eq:baz_MB}
      \underbrace{\frac{\partial w}{\partial t}}_{\text{(a)}} = \underbrace{\vphantom{ \left(\frac{a}{b}\right) } - \nabla \cdot \vec{J}_{Darcy}}_{\text{(b)}} + \underbrace{\frac{\partial w_d}{\partial t}}_{\text{(c)}}
    \end{equation}

    O termo $(a)$ corresponde a taxa de liberação de água livre do sistema em um
    dado ponto, sendo portanto equivalente à quantidade de água transportada
    para (ou de) tal ponto, representado pelo termo (b), e a quantidade de água liberada durante a
    desidratação (ou consumida durante a hidratação), termo (c).

    O balanço energético por sua vez se dá através da igualdade da taxa de
    consumo (ou liberação de energia térmica) em dado ponto, termo $(a)$, com a taxa de
    consumo de energia para a liberação da água livre, termo $(b)$, a quantidade
    de calor transportada por convecção, termo $(c)$ e a quantidade de calor
    transportado por condução, termo $(d)$.

    \begin{equation}
      \label{eq:baz_EB}
      \underbrace{\rho \ C_p \ \frac{\partial T}{\partial t}}_{\text{(a)}} = \underbrace{C_a \ \frac{\partial w}{\partial t}}_{\text{(b)}} + \underbrace{\vphantom{\left(\frac{a}{b}\right)} C_w \vec{J}_{Darcy}\cdot \nabla T}_{\text{(c)}} - \underbrace{\vphantom{\left(\frac{a}{b}\right)} \nabla \cdot \vec{q}_{Fourier}}_{\text{(d)}}
    \end{equation}
    
    Os trabalhos de Ba\v{z}ant também trouxeram grandes avanços em se tratando
    das propriedades dos concretos. Devido as inúmeras transformações na
    microestrutura do material decorrentes de seu aquecimento, quase todas as
    propriedades se tornam funções da temperatura e da umidade relativa (e
    portanto da pressão). As mudanças mais dramáticas são na permeabilidade do
    material bem como nas curvas de sorção que definem o teor de água livre no
    sistema. Portanto, no modelo descrito na presente seção, tanto a densidade,
    quanto a condutividade térmica, o calor sensível do concreto e da água serão
    considerados constantes, sendo apenas a permeabilidade, as curvas de sorção
    isotérmica e a água liberada por desidratação funções da temperatura e/ou
    umidade relativa (pressão).

    \subsubsection{Curvas de Sorção Isotérmica}
    Através de relações semi-empíricas Ba\v{z}ant descreve as curvas de sorção
    como uma função definida em dois regimes distintos, a região na qual o
    concreto está insaturado de água (umidade relativa menor que 1, $h(P, T) \leq 0.96$)
    e a região super saturada ($h{P,T} \geq 1.04$), além disso define-se uma região de
    intervalo que interpola linearmente tais valores:
    \begin{equation}
      \label{eq:baz_phi}
      w(P, T) =
      \begin{cases} 
      w_c \left( \frac{w_0}{c} \ h(P,T) \right)^{\frac{1}{m(T)}} & h(P, T)\leq 0.96 \\
      w_{0.96} + (h(P, T) - 0.96) \frac{(w_{1.04} - w_{0.96})}{(1.04-0.96)} & 0.96 < h(P, T) < 1.04 \\
      (1 + 3 \ \epsilon^v)\frac{n}{v} & 1.04\leq h(P, T) 
      \end{cases}
    \end{equation}
    
    Onde na região insaturada ($h(P, T) \leq 0.96$) $w_c$ é a quantidade de
    cimento em $kg$ por $m^3$ de concreto, $w_0$ é a quantidade inicial de água em $kg$
    por $m^3$ de concreto e $\frac{1}{m(T)}$ é uma correlação empírica que
    envolve a dependência da tensão superficial da água com a temperatura, bem
    como quaisquer erros experimentais das medidas e das simplificações
    realizadas.

    Segundo Ba\v{z}ant, tal correlação foi obtida a partir de considerações
    termodinâmicas em um modelo simplificado de um concreto com poros de geometria
    constante e quantidade de água adsorvida negligenciável. Tal modelo concluiu
    que a variação da quantidade de água livre nos concretos insaturados segue
    uma lei de potências até o ponto de saturação.

    Na definição das curvas de sorção, no intervalo de saturação, $w_{0.96}$ e
    $w_{1.04}$ são funções da temperatura dos valores de água livre no sistema
    em umidades relativas, $h(P, T)$, iguais a 0.96 e 1.04, respectivamente.
    Observe que tal correlação garante apenas uma continuidade das derivadas da
    curva de sorção por partes. Tal ponto é abordado por outros autores que
    desenvolveram formas de forçar tanto a continuidade das curvas de sorção bem
    como de suas derivadas.

    Por fim, na região super saturada, $\epsilon$ representa a enquanto $n$ é a
    porosidade e $v$ é o volume especifíco da água. Conforme reportado por
    Ba\v{z}ant, a quantidade de água poderia ser assumida a partir de tabelas
    com propriedades termodinâmicas da água através do volume dos poros do
    material e da pressão e temperatura em dado momento, porém, os cálculos
    usando tal procedimento resultavam pressões da ordem de 1000 atm, muito
    maiores do que os valores esperados. A explicação para essa divergência vem
    do fato de que o volume dos poros é uma função crescente da temperatura
    devido dois efeitos sobrepostos, sendo a desidratação do cimento, o que
    aumenta o volume dos poros e principalmente a redução da quantidade de água
    adsorvida nas paredes dos poros o que aumenta o volume livre do sistema.

    \subsubsection{Permeabilidade}
    O tratamento dado a permeabilidade por Ba\v{z}ant justifica a sua hipótese
    de que o transporte de água no material se dá por um único fluído que
    representa o vapor de água, o líquido e a água adsorvida. Isto se dá pelo
    fato de que, especialmente em temperaturas inferiores a 100$^{\circ}$, a
    capilaridade interna do material não é totalmente contínua apresentando
    empescoçamentos que controlam a permeabilidade efetiva do material
    \cite{bazant1978}.

    Portanto, em baixas temperaturas (abaixo de 95$^{\circ}$) a energia de ativação para a migração de
    água nas camadas de água adsorvida na região de empescoçamento ($f_2(T)$) e o
    transporte de umidade nessa região controlam ($f_1(h)$) a permeabilidade do material
    conforme descrito pela equação \ref{eq:baz_perm}.
    
    \begin{equation}
      \label{eq:baz_perm}
      K(P, T) =
      \begin{cases} 
       K_0 f_1(h) f_2(T) & T \leq 95 \\
       K_0 f_2(95) f_3(T) & T > 95 
      \end{cases}
    \end{equation}

    Onde,

    \begin{equation}
      \label{eq:f1}
     f_1(h) = \alpha + \frac{1-\alpha}{1+\left(\frac{1-h}{1-h_t}\right)^4} 
    \end{equation}

    e $\alpha$ representa a largura das regiões de empescoçamento como uma
    função linear da temperatura (sendo que varia de 0.05 em temperatura
    ambiente e 1 à 95$^{\circ}$)e $h_t$ é a umidade de transição (equivalente a
    0.75). Além disso, define-se $f_2(T)$ como uma equação do tipo Arhenius:
    \begin{equation}
      \label{eq:f2}
     f_2(T)  = \exp{\left[\frac{Q}{R}\left( \frac{1}{T_0} - \frac{1}{T} \right) \right]}
    \end{equation}
    
    
    Em temperaturas superiores a $95^{\circ}$ ainda considera-se o efeito da
    energia de ativação para a temperatura de $95^{\circ}$ além de se considerar
    uma função que gere o aumento de duas ordens de grandeza na permeabilidade
    decorrente da transição do regime regido pela energia de ativação do
    transporte de água adsorvida na região do pescoço para um
    controlado pela viscosidade da mistura de água líquida e vapor de água
    conforme descreve $f_3{T}$:
    \begin{equation}
      \label{eq:f3}
     f_3(T) = \exp{\left( \frac{T-95}{0.881+0.214 \ (T-95)} \right)}
    \end{equation}

    Onde as constantes numéricas foram determinadas pela interpolação de dados
    experimentais.

    Considerando as variáveis dependentes da temperatura e pressão, a quantidade
    de água químicamente ligada liberada vem da interpolação de dados
    experimentais de ensaios termogravimétricos realizados em amostras pré
    secadas. Com o sistema de equações parciais diferenciais, e as propriedades
    dos materiais, define-se uma geometria e condições de contorno, escolhe-se
    um método numérico tendo assim resultados dos campos de pressão e
    temperatura em função do tempo. Ba\v{z}ant também trouxe contribuições
    no desenvolvimento de adequações da metodologia dos elementos
    finitos o que ampliava a performance dos programas computacionais.

    Nos anos seguintes, diferentes autores desenvolveram e ampliaram as
    capabilidades do modelo de Ba\v{z}ant tanto nas mesmas aplicações
    previamente abordadas quanto em outras áreas como a de secagem de concretos
    refratários.
    
    
    \subsection{Modelos derivados de Ba\v{z}ant}\label{sec:deriv_bazant}
    A presente seção introduz apenas superficialmente os modelos derivados dos
    trabalhos de Ba\v{z}ant, portanto questões específicas estão fora do escopo
    desta seção. Para maiores detalhes referencia-se aos trabalhos dos
    respectivos autores. Tal decisão justifica-se pelo fato de que nesta parte
    busca-se traçar uma visão geral do desenvolvimento dessa área, além disso
    aspectos dos modelos utilizados serão mais detalhados na seção Materiais e
    Métodos.

    \subsubsection{Modelo de Gong}
    O modelo utilizado no presente trabalho baseia-se no modelo de Gong
    \cite{Gong1995a} que por sua vez é extremamente similar ao de Ba\v{z}ant
    distinguindo-se apenas pelas propriedades utilizadas e condições de controno
    empregadas uma vez que a aplicação do modelo visa a simulação da secagem de
    concretos refratários. Tal modelo será descrito em maiores detalhes na seção
    \ref{metodologia}.
    
    
    \subsubsection{Modelos de Gawin}
    O modelo de Gawin\cite{Gawin1999}  foi resultado da intensa interação de distintos grupos de
    pesquisas Europeus dentro do escopo do projeto Eurocode \cite{Eurocode} que
    visa a melhoria da segurança de estruturas da construção civíl sujeitas a
    incêndios.

    O modelo de Gawin diferentemente do de Ba\v{z}ant considera o fluxo
    multifásico dentro do material separando o balanço da massa em balanço de
    ar, vapor de água e água líquida. Além disso, propriedades como
    condutividade térmica, densidade e calor específico são considerados como
    funções da tempratura e/ou da pressão.

    Por fim, além dos aspectos termohídricos engloba-se o efeito termomecânico
    resultando no modelo termohigromecânico que prevê os estados de tensão bem
    como o desenvolvimento de danos no material sujeito à altas temperaturas.

    Como um contraponto, tal modelo se torna altamente complexo com baixo apelo
    tecnológico para a simulação de sistemas distintos uma vez que para um único
    material mais de 50 parâmetros precisam ser medidos e/ou estimados.
    
    \subsubsection{Modelos de Davie}
    O modelo de Davie é similar ao de Gawin se diferenciando pelo fato de que
    utiliza um modelamento termomecânico mais simplista além de simplificar o
    modelo ao descartar a influência do transporte de energia por convecção no
    material.
    
    \subsubsection{Modelo de Ben\v{e}s}
    Ben\v{e}s et al desenvolveram análises numéricas trazendo desenvolvimentos
    do ponto de vista matemático como a confirmação de que tal problema
    matemático de fato apresenta uma solução única e real \cite{Benes2013a}.
    Entretanto o modelo em questão é mais simplista similar ao do Gong com a
    única diferença de que engloba a contribuição do calor de desidratação
    necessário para liberar a água fisicamente ligada às fases do cimento.
    
    \subsubsection{Modelo de Fey}
    Por fim o modelo de Fey \cite{Fey2016b} é o mais recente trabalho referente à secagem de
    materiais refratários. É um modelo intermediário ao de Gawin e do Ba\v{z}ant
    no sentido de que é multifásico porém não considera os efeitos
    termomecânicos. Sua maior contribuição é portanto em relação às análises e
    metodologias propostas em como otimizar a curva de secagem baseado nos
    resultados da simulação.

    A Tabela \ref{tab:comp_models} resume a comparação dos modelos derivados do
    modelo de Ba\v{z}ant.


    
    \afterpage{%
    \clearpage% Flush earlier floats (otherwise order might not be correct)
    \begin{landscape}% Landscape page
    \hskip-4.0cm
\begin{tabular}{lllllll}
      \label{tab:comp_models}
Modelo   &                                                          & Gong & Benes & Fey      & Tenchev   & Gawin     \\
Geral    & Fases Consideradas                                       & g    & g     & s + l +g & s + l + g & s + l + g \\
         & Número de componentes no gás                             & 1    & 1     & 2        & 2 ($P_g=P_l$) & 2         \\
         & Número de fases de água                                  & 1    & 1     & 3        & 3         & 3         \\
         & Dimensões                                                & 1D   & 2D/3D & 1D/2D    & 1D/2D     & 1D/2D     \\
Térmico  & Condução de calor ($\nabla \cdot (k \nabla T)$)                            & S    & S     & S        & S         & S         \\
         & Convecção de calor ($C_w K/g \nabla P \cdot \nabla T$)
                                                                     & S    & S     & S        & N         & S         \\
         & Calor latente de vaporização/condensação ($C_a$)           & S    & S     & S        & S         & S         \\
         & Calor latente de desidratação ($h_d$)                      & N    & S     & S        & S         & S         \\
Hídrico  & Difusão de vapor de água ($\nabla· (D \nabla w)$)                     & N    & N     & S        & S         & S         \\
         & Advecção de fluídos ($\nabla \cdot (K/g \nabla P)$)                        & S    & S     & S        & S         & S         \\
         & Água fisicamente ligada                                  & N    & N     & N        & S         & S         \\
         & Água liberada por desidratação (dwd/dt)                  & S    & S     & S        & S         & S         \\
         & Água liberada decorrente da mudança de fases             & N    & N     & S        & S         & S         \\
Químico  & Desidratação                                             & N    & N     & N        & N         & S         \\
Mecânico & Degradação da rigídez induzida pela temperatura          & N    & N     & N        & N         & S         \\
         & Degradação da regídez induzida pela solicitação mecânica & N    & N     & N        & S         & S         \\
         & Tensões residuais                                        & N    & N     & N        & S         & S         \\
         & Tensões térmicas transientes                             & N    & N     & N        & S         & S        
\end{tabular}
 
        \captionof{table}{Comparação dos modelos derivados de Ba\v{z}ant.}% Add 'table' caption
    \end{landscape}
    \clearpage% Flush page
    }
   

    
%%% Local Variables:
%%% mode: latex
%%% TeX-master: "TCC-Secagem"
%%% End:

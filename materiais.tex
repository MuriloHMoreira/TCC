A proposta deste trabalho é desenvolver um sistema de caracterização de espumas baseado em medidas da fração de líquido e altura da espuma com o tempo. Para isso, foram utilizados sensores resistivos de umidade do solo (YL-69) e um sensor ultrassônico (HC-SR04). O desenvolvimento do projeto pode ser dividido em 5 partes:

\begin{enumerate}[label=\itshape\roman*]

\item \textit{Desenvolvimento do circuito:} acoplamento de todos os sensores às portas do Arduino, com o uso de uma \textit{protoboard} e \textit{jumpers}, e do Arduino ao computador para alimentação, carregamento do \textit{software} e coleta de dados;

\item \textit{Desenvolvimento do software:} programação de um código capaz de controlar o \textit{hardware} do sistema, coletando os dados dos sensores e realizando os cálculos necessários;

\item \textit{Construção do protótipo:} foi utilizado um tubo de PVC com alguns entalhes para encaixar os sensores de umidade do solo;

\item \textit{Realização de testes:} algumas espumas foram caracterizadas com o proposto sistema, a fim de validar suas funcionalidades;

\item \textit{Análise de resultados:} os resultados dos testes realizados na quarta etapa foram analisados e interpretados.

\end{enumerate}

        
\section{Caracterização Experimental}\label{mat:exp}
Teste 1

    \subsection{Porosidade Aparente}\label{mat:porosidade}
    Teste a

    \subsection{Permeabilidade}\label{mat:perm}
    Teste b

    \subsection{Resistência Mecânica}\label{mat:rm}
    Teste c

    \subsection{TGA}\label{mat:TGA}
    Teste d

    \subsection{Condutividade Térmica}\label{mat:condutividade}
    Teste e


\section{Desenvolvimento do modelo em FEniCS}\label{mat:fenics}
Teste 2

    \begin{lstlisting}
    digitalWrite(sensorVCC1, HIGH);
    delay(10000);
    sensorValue1 = analogRead(sensorPin1);
    digitalWrite(sensorVCC1, LOW);
    Serial.print(sensorValue1);

    digitalWrite(trigger, LOW);
    delayMicroseconds(2);
    digitalWrite(trigger, HIGH);
    delayMicroseconds(10);
    digitalWrite(trigger, LOW);
    delayMicroseconds(2);
    time=pulseIn(echo, HIGH);
    distance = time*340/20000;

    delay(600000);
    \end{lstlisting}

    
    \subsection{Geometria e Condições de Contorno}\label{mat:geo_bcs}
    Teste 3

    \subsection{Sistema de Equações}\label{mat:eqs}
    Teste 4

    \subsubsection{Forma Forte}\label{mat:forte}
    Teste a

    \subsubsection{Forma Fraca}\label{mat:fraca}
    Teste a

    \subsection{Estrutura do script}\label{mat:script}
    Teste 5

    \subsection{Pós-processamento}
    Teste 6
\section{Secagem de Refratários Monolíticos}\label{secagem}
De todas as etapas de processamento de materiais monolíticos com ligantes
hidráulicos, a secagem é uma das que mais toma tempo durante o processo de
reparo e revestimento de equipamentos \cite{da2015refractory}. Devido a tal
fator, há um grande potencial de ganho (em termos de tempo de reparo e de
consumo de energia) na utilização de procedimentos de secagem mais eficientes.
Porém, qualquer dano provocado no material durante tal etapa compromete a vida
útil do refratário, sendo necessário um controle que balanceie tais cenários.

Além de tais desafios, garantir que a secagem, de fato, siga os procedimentos
recomendados pelo produtor (as curvas de secagem) é algo complexo dado
ineficiências do sistema de aquecimento (outro setor onde a simulação
computacional pode proporcionar grandes ganhos referentes a otimização do
sistema de aquecimento, bem como maior precisão e acurácia do sistema
controlador de temperaturas) e da falta de instrução dos operadores
\cite{da2015refractory}. Soma-se o fato de que muitas vezes as recomendações
dadas pelo produtor são motivadas muito mais pelo temor de ter que arcar com os
custos de uma explosão ou de danos ao equipamento devido problemas na secagem do
que, de fato, fundamentos científicos e testes experimentais.

Sendo assim há três estratégias comuns para a redução do risco de danos durante
o processo de secagem \cite{da2015refractory}:

\begin{enumerate}
\item Otimização da curva de secagem: \\ Para que a maior quantidade de água
  seja removida durante o regime de evaporação (25$^{\circ}$C a 100$^{\circ}$C)
  e não durante o regime de ebulição (100$^{\circ}$C a 300$^{\circ}$C);
\item Alteração da microestrutura do material: \\ Para aumentar a permeabilidade e
  diminuir o nível de pressurização no interior do material;
\item Aumento da resistência mecânica: \\ Para que o material resista às tensões
  triaxiais decorrente da pressão do vapor de água.
\end{enumerate}

As próximas seções abordam cada uma dessas estratégias de forma mais
compreensiva, a fim de mostrar o estado da arte e como as simulações
computacionais poderiam complementar os estudos nesta área.

\subsection{Curvas de secagem}
Uma forma bastante comum no meio industrial para se definir uma curva de secagem
é apresentado na Tabela \ref{tab:industrial_HUC}.

\begin{table}[ht]
\centering
\caption{Procedimento de secagem recomendado pela empresa AGC Ceramics.
  Adaptado de \cite{agc2016}.}
\label{tab:industrial_HUC}
\begin{tabularx}{\textwidth}{|X|X|X|X|X|}
\hline
Produto & Concretos com baixo teor de cimento para a indústria de alumínio & Concretos com baixo teor de cimento para outras indústrias & Concretos convencionais \\ \hline
Da temperatura ambiente até 200°C & 50°C/h & 50°C/h & 50°C/h \\ \hline
Patamar em 200°C & 1h/10mm de espessura & 1h/20mm de espessura & 1h/20mm de espessura \\ \hline
De 200°C até 350°C & 25°C/h & 25°C/h & 50°C/h \\ \hline
Patamar em 350°C & 1h/10mm de espessura & 1h/20mm de espessura & 1h/20mm de espessura \\ \hline
De 350°C até a temperatura de trabalho & 25°C/h & 50°C/h & 50°C/h \\ \hline
\end{tabularx}
\end{table}

Observa-se que as taxas de aquecimento são definidas em diferentes intervalos de
temperatura baseados aproximadamente na decomposição dos hidratos presentes no
material. Além disso, o período de tempo dispendido em cada etapa é definido a
partir da espessura do revestimento, como uma maneira de se levar em
consideração o efeito da distribuição de temperatura no interior do material.

A crítica que se faz de tal procedimento é que tal correlação linear entre a
temperatura e a espessura do componente não condiz os resultados experimentais e
numéricos. Ademais, há a influência do próprio transporte de massa no interior
do revestimento cerâmico nos perfis de temperatura, o que pode gerar perfis
completamente distintos.

Tais orientações deveriam ser complementadas por resultados experimentais e
simulações numéricas, levando em conta não só as dimensões do refratário como
também as condições de contorno, a geometria, a condutividade térmica e a
permabilidade do material.

Do ponto de vista empírico, ensaios de análise termogravimétrica podem ser de
grande importância para verificar se os intervalos de transformações propostos
na curva de secagem, de fato correspondem com as reações de desidratação.

\subsection{Aditivos para secagem}
As recomendações 2 e 3 (alteração da microestrutura e aumento da resistência
mecânica) podem ser implementadas através de aditivos adicionados na composição
dos concretos. Dois casos específicos serão apresentados, são eles o uso de
fibras poliméricas e fibras metálicas. Porém é importante salientar que inúmeras
outras possibilidades também podem contribuir para o aumento de permeabilidade
ou o aumento da resistência mecânica dos refratários (como uso de diferentes
sistemas ligantes, ou fases estabilizadas).

As fibras metálicas são capazes de ampliar a energia de fratura do concreto ao
promover diferentes mecanismos de tenacificação como {\it crack-bridging}, {\it
  microcracking} e o fenômeno de {\it pullout} \cite{da2015refractory}. Dessa
maneira, há uma maior resistência ao dano por parte do material, de modo que as
tensões devido a pressão do vapor não sejam capazes de promover o crescimento
catastrófico das trincas. Além do efeito mecânico, o {\it microcracking}
decorrente do {\it mismatch} dos coeficientes de expansão da matriz e das fibras
metálicas, pode promover um aumento local da permeabilidade do material conforme
reportado por Li et al \cite{li2019}.

Por outro lado, as fibras poliméricas não apresentam quaisquer efeitos de
tenacificação, inclusive promovendo a formação de defeitos que podem diminuir a
resistência mecânica do material, uma vez que estas sofrem decomposição em
baixas temperaturas (200$^\circ$C a 300$^\circ$C) e promovem o aumento da
permeabilidade do material.

Novamente, o uso da simulação computacional se faz importante uma vez que é
necessário identificar a temperatura na qual a pressão no interior do material é
máxima para selecionar o {\it grade} correto de polímero (como as fibras de
polipropileno) que apresente uma temperatura de decomposição coerente.

Dessa forma justifica-se a busca por modelos númericos que possam garantir a
otimização das curvas de secagem seja como um complemento às metodologias já
sugeridas (como otimização da taxa de aquecimento, aumento da permebailidade ou
aumento da resistência mecânica) ou ainda através de novas estratégias
descobertas através da possibilidade de se obter os campos de pressão e
temperatura no interior do material durante o fenômeno de secagem. Portanto, a
próxima seção traz uma breve revisão dos modelos existentes que descrevem o
transporte de massa e de energia de materiais sujeitos à altas
temperaturas.
     
        
%%% Local Variables:
%%% mode: latex
%%% TeX-master: "TCC-Secagem"
%%% End:
